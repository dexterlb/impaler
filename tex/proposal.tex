\documentclass[main.tex]{subfiles}
\begin{document}

The subject of this proposal is a programming language (and implementation)
that is expressive enough for general-purpose usage, but also usable
in constrained environments like microcontrollers and video games with little
overhead.

\subsection{Informal introduction}

For a long time, I've wanted to build a (functional) programming language that
combines the following features that I like in other programming languages:

\begin{mitemize}
\item LISP-like syntax
\item Full immutability and reliance on persistent data structures
\item First-class CPS
\item Controlled partial evaluation
\end{mitemize}

While focusing on the features outlined above, I would like to completely
neglect some aspects that are standard in most serious programming languages:

\begin{mitemize}

\item \note{Stoyan did the same with C(M) and it seems to work}No means of
constructing cyclic data structures: full immutability means that there is no
way to construct doubly-linked lists, boxed-element queues, etc - when needed,
\reword{cyclicism} may be emulated by using indices in persistent arrays

\item No overly-complex garbage-collection mechanism: since there are no
cyclic data structures, simple reference counting will suffice. While
optimising GC performance is not a priority, making use of the partial
evaluator to reduce reference count increment-decrement chains
(as defined in \cite{ms}) may be worth researching.

\end{mitemize}

A lot of this work is inspired by AnyDSL\cite{anydsl}, namely the Impala
programming language, which focuses on controlled partial evaluation.

In fact, a minimum working implementation of \reword{my language} may consist
of a toy interpreter written in Impala, taking care that the partial evaluator
would specialise the interpreter for a given program (\reword{first Futamura
projection}), generating efficient target code \reword{stripped from
abstractions}.

\subsection{More formal exposition}

The initial goals for the language are to:
\begin{menumerate}
    \item\label{functional} be functional
    \item\label{immutable} have no mutable data
    \item\label{metaprogramming} have powerful, LISP-like metaprogramming
    \item\label{allows-async} allow implementation of features like asynchronous
    execution and message-passing by user libraries instead of requiring
    extra language features
    \item\label{simple} have a very small core that makes it trivial to
    write an interpreter that successfully executes all programs written
    in the language, and to target different platforms
    \item\label{performant} have an implementation (i.e. compiler) that
    focuses on generating performant (with higher priority on low latency rather
    than high throughput) target code
\end{menumerate}

\menumref{allows-async} may be addressed by having first-class CPS.
If arbitrary functions can access their \emph{return} continuation
and pass it around, asynchronous code may be easily implemented.
This design decision has been made in the Impala programming
language\cite{anydsl}, allowing for superior ergonomics.

\menumref{metaprogramming} implies heavy use of abstractions. In order for the
generated code to be reasonably performant \menumref{performant}, the compiler
must be able to generate abstraction-free specialised code. Employing PE for
specialising generic code has proven\cite{anydsl} to be effective. I want to
explore whether this can be taken a step further, by making macros live in the
same domain as functions. This way, there is no need for a macro expansion
phase: instead, macros would be `eaten' by PE, since their arguments are
constant code expressions.

\subsection{Use Case: Embedded Programming}

Many scenarios for embedded programming are expressed using
state machines, whose transitions are influenced by interrupt events.

However, a state machine is difficult to reason about: what we really
want is to write sequential code that is under-the-scenes converted
to a state machine. Building such a DSL is trivial with CPS (or other
similar but weaker constructions like coroutines or async/await).

Example: a \textbf{sleep} function that is usable within an event loop's body,
but under the scenes causes the body to return a special value (containing the
\textbf{sleep}'s callsite continuation), so that the event loop logic may call
said continuation in response to a timer event, to resume execution from the
point in the code after the callsite

\pagebreak

I want to build a programming language with the following properties:
\begin{enumerate}
    \item very small and simple
    \item powerful metaprogramming using lisp-like macros
    \item completely immutable
    \item performant
    \item \emph{compiled} (as much as this term may apply to lisp-like languages)
\end{enumerate}

The main technique I wish to leverage is \emph{partial evaluation}, which is
inevitably used in any sufficiently mature implementation of a programming
language compiler/interpreter, but usually not as a first-class concept

\reword{I want to do the following things which would normally be slow} and
leverage partial evaluation to attempt to make performance on-par with
\reword{SOTA} solutions:
\begin{itemize}
    \item full immutability
    \item runtime function specialisation (or maybe not, this might not be feasible)
    \item reference-counted garbage collection
\end{itemize}

Most languages in the wild [citation needed] are either \ref{fixme, simple} or
\ref{fixme, performant}. This can be addressed with partial evaluation.

Also, most languages [citation needed, erlang] are either immutable or performant.
This can be addressed by using persistent data structures.



\greenbox{

\textbf{BULLSHIT ALERT! The assumptions in this paragraph are stupid,
    uninformed and outright wrong! Do your fact-checking, past me!}

In lisp-like languages, there are usually two phases: first we execute macros,
then we execute code. With compiled lisp-like languages, macro evaluation is
usually done during compilation. This creates problems - for example, using
`runtime' code inside macros is usually not straight-forward.  I wish to
implement a small lisp-like language to see if these two stages can be merged
by having macros be simply functions with slightly different semantics -
essentially, performed during runtime, from the perspective of the programmer.
}

This document is an informal description of the language. The examples will
have generic `lispy' pseudocode notation.

The language features could be defined in several tiers:
\begin{itemize}
    \item Tier 0 (\emph{proof of concept})
    \item Tier 1 (\emph{toy language})
    \item Tier 2 (\emph{practically usable})
\end{itemize}

\subsection{Proposed proof of concept language (Tier 0)}

Here we'll specify a programming language.

Semantically, our language shall have the following classes of internal values:
\begin{enumerate}
    \item symbols
    \item lists
    \item opaque external values
    \item function objects
\end{enumerate}

The only classes of internal values that may be expressed syntactically are
symbols and lists\footnote{%
    It would be quite beneficial for our language to
    support other atoms apart from symbols, namely numbers and strings.
    However, for a proof of concept they are not needed: consider that numbers
    and chars are simply symbols, while strings are lists of chars.}.
External values and function objects are created by the interpreter/compiler as
results of evaluating complex expressions.

Semantics shall be defined as such:

\begin{itemize}
    \item abstraction: \code{(lambda (\(a_1 a2 ... a_n\)) \(P\))} - evaluates to
        a function object which carries the current scope (environment)
    \item application: \code{(f \(S_1 S_2 ... S_n\))} - evaluates the function
        object associated with the symbol \code{f} in the current scope,
        using the results from evaluating \(S_1, S_2, ..., S_n\) as arguments.
    \item quote: \code{(quote \(S_1 S_2 ... S_n\))} - evaluates to the literal
        list \code{(\(S_1 S_2 ... S_n\))}
    \item expansion: \code{(expand \(S\))} - syntactically replace this expression
        with the result from evaluating \(S\), and then evaluate it
\end{itemize}

\subsection{Garbage collection}
foo

\subsection{Partial evaluation}

In this context, we regard \emph{partial evaluation} to be ...

\begin{lstwrap}\begin{lstlisting}[language=lisp]
(here is some code)
\end{lstlisting}\end{lstwrap}

\begin{mexample}
    this shall be an example
    \begin{lstwrap}\begin{lstlisting}[language=lisp]
    (here is some code inside the example)
    \end{lstlisting}\end{lstwrap}
\end{mexample}

\subsection{to be continued}
still unfinished \pika

\end{document}
