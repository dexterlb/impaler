\documentclass[main.tex]{subfiles}
\begin{document}

The subject of this proposal is a programming language (and implementation)
that is expressive enough for general-purpose usage, but also usable
in constrained environments like microcontrollers and video games with little
overhead.

The initial goals for the language are to:
\begin{menumerate}
    \item\label{functional} be functional
    \item\label{immutable} be immutable (i.e. no mutable data structures)
    \item\label{metaprogramming} have powerful metaprogramming
    \item\label{allows-async} allow implementation of features like asynchronous
    execution and message-passing by user libraries instead of requiring
    extra language features
    \item\label{simple} have a very small core that makes it trivial to
    write an interpreter that successfully executes all programs written
    in the language, and to target different platforms
    \item\label{performant} have an implementation (i.e. compiler) that
    focuses on generating performant (low latency more importantly than
    high throughput) target code
\end{menumerate}

\menumref{metaprogramming} may be addressed by having LISP-like syntax
and macros.

\menumref{allows-async} may be addressed by having first-class CPS.
If arbitrary functions can access their \emph{return} continuation
and pass it around, asynchronous code may be easily implemented.

\menumref{performant} may be addressed by implementing a compiler that leverages
aggressive partial evaluation in order to inline the extra calls caused by
metaprogramming and generic programming.

\subsection{Use Case: Embedded Programming}

Many scenarios for embedded programming may be expressed using
the following constructs:
\begin{mitemize}
    \item State machines, some of whose transitions are influenced
    by interrupt events
    \item bla
\end{mitemize}

Example: a \textbf{sleep} function that is usable within an event loop's body,
but under the scenes returns a special value (containing the \textbf{sleep}'s callsite
continuation) from the event loop that causes the loop to be called again
in response of a specific interrupt, to resume execution from the point after the callsite

\pagebreak

I want to build a programming language with the following properties:
\begin{enumerate}
    \item very small and simple
    \item powerful metaprogramming using lisp-like macros
    \item completely immutable
    \item performant
    \item \emph{compiled} (as much as this term may apply to lisp-like languages)
\end{enumerate}

The main technique I wish to leverage is \emph{partial evaluation}, which is
inevitably used in any sufficiently mature implementation of a programming
language compiler/interpreter, but usually not as a first-class concept

\reword{I want to do the following things which would normally be slow} and
leverage partial evaluation to attempt to make performance on-par with
\reword{SOTA} solutions:
\begin{itemize}
    \item full immutability
    \item runtime function specialisation (or maybe not, this might not be feasible)
    \item reference-counted garbage collection
\end{itemize}

Most languages in the wild [citation needed] are either \ref{fixme, simple} or
\ref{fixme, performant}. This can be addressed with partial evaluation.

Also, most languages [citation needed, erlang] are either immutable or performant.
This can be addressed by using persistent data structures.



\greenbox{

\textbf{BULLSHIT ALERT! The assumptions in this paragraph are stupid,
    uninformed and outright wrong! Do your fact-checking, past me!}

In lisp-like languages, there are usually two phases: first we execute macros,
then we execute code. With compiled lisp-like languages, macro evaluation is
usually done during compilation. This creates problems - for example, using
`runtime' code inside macros is usually not straight-forward.  I wish to
implement a small lisp-like language to see if these two stages can be merged
by having macros be simply functions with slightly different semantics -
essentially, performed during runtime, from the perspective of the programmer.
}

This document is an informal description of the language. The examples will
have generic `lispy' pseudocode notation.

The language features could be defined in several tiers:
\begin{itemize}
    \item Tier 0 (\emph{proof of concept})
    \item Tier 1 (\emph{toy language})
    \item Tier 2 (\emph{practically usable})
\end{itemize}

\subsection{Proposed proof of concept language (Tier 0)}

Here we'll specify a programming language.

Semantically, our language shall have the following classes of internal values:
\begin{enumerate}
    \item symbols
    \item lists
    \item opaque external values
    \item function objects
\end{enumerate}

The only classes of internal values that may be expressed syntactically are
symbols and lists\footnote{%
    It would be quite beneficial for our language to
    support other atoms apart from symbols, namely numbers and strings.
    However, for a proof of concept they are not needed: consider that numbers
    and chars are simply symbols, while strings are lists of chars.}.
External values and function objects are created by the interpreter/compiler as
results of evaluating complex expressions.

Semantics shall be defined as such:

\begin{itemize}
    \item abstraction: \code{(lambda (\(a_1 a2 ... a_n\)) \(P\))} - evaluates to
        a function object which carries the current scope (environment)
    \item application: \code{(f \(S_1 S_2 ... S_n\))} - evaluates the function
        object associated with the symbol \code{f} in the current scope,
        using the results from evaluating \(S_1, S_2, ..., S_n\) as arguments.
    \item quote: \code{(quote \(S_1 S_2 ... S_n\))} - evaluates to the literal
        list \code{(\(S_1 S_2 ... S_n\))}
    \item expansion: \code{(expand \(S\))} - syntactically replace this expression
        with the result from evaluating \(S\), and then evaluate it
\end{itemize}

\subsection{Garbage collection}
foo

\subsection{Partial evaluation}

In this context, we regard \emph{partial evaluation} to be ...

\begin{lstwrap}\begin{lstlisting}[language=lisp]
(here is some code)
\end{lstlisting}\end{lstwrap}

\begin{mexample}
    this shall be an example
    \begin{lstwrap}\begin{lstlisting}[language=lisp]
    (here is some code inside the example)
    \end{lstlisting}\end{lstwrap}
\end{mexample}

\subsection{to be continued}
still unfinished \pika

\end{document}
