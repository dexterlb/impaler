\documentclass[11pt]{amsart}

\usepackage{amsfonts}
\usepackage{amssymb}
\usepackage{graphicx,color}
% \usepackage{a4wide}
\usepackage{amsmath}
\usepackage{amsthm}

\textwidth 14.8cm \textheight 19.5cm \topmargin 0in
\oddsidemargin 0.5in
\evensidemargin 0.5in
\parskip 1mm
\def\ds{\displaystyle}

\title[Title]{\bf A functional programming language designed for easy partial evaluation}

\author[Angel]{Angel Angelov}
% \address{Baba \\
% foo
% }
\email{angel@qtrp.org}

% \date{\today}

\newtheorem{theorem}{Theorem}[section]
\newtheorem{lemma}[theorem]{Lemma}
\newtheorem{corollary}[theorem]{Corollary}
\newtheorem{proposition}[theorem]{Proposition}
\newtheorem{conjecture}[theorem]{Conjecture}
\theoremstyle{definition}
\newtheorem{defn}[theorem]{Definition}
\newtheorem{example}[theorem]{Example}
\newtheorem{remark}[theorem]{Remark}

\begin{document}

\maketitle

\begin{center}
Faculty of Mathematics and Informatics, Sofia University
\end{center}

\section*{Abstract}

I present a programming language that loosely resembles Scheme, while
differing on a few key points:
\begin{itemize}
    \item foo
\end{itemize}

% \begin{thebibliography}{99}
%
% \bibitem{BS} E. Bannai, N. J. A. Sloane, Uniqueness of certain spherical codes, {\it Can. J. Math.} 33, 1981, 437-449.
%
% \bibitem{CS} {	J.H.Conway, N.J.A.Sloane}, {\it Sphere Packings, Lattices and Groups}, Springer -- Verlag, New York 1988.
%
% \bibitem{Sze}
% G.\,Szeg\H{o}, \emph{Orthogonal Polynomials},
% Amer. Math. Soc. Col. Publ., { 23}, Providence, RI, 1939.
%
% \end{thebibliography}

\bigskip
\end{document}
