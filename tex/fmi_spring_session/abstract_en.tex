\documentclass[11pt]{amsart}

\usepackage{amsfonts}
\usepackage{amssymb}
\usepackage{graphicx,color}
% \usepackage{a4wide}
\usepackage{amsmath}
\usepackage{amsthm}

\textwidth 14.8cm \textheight 19.5cm \topmargin 0in
\oddsidemargin 0.5in
\evensidemargin 0.5in
\parskip 1mm
\def\ds{\displaystyle}

% margin note
\newcommand\mnote[1]{\leavevmode\marginpar{\scriptsize{ #1 }}}
\newcommand\note[1]{\mnote{\textcolor{orange}{ #1 }}}
% margin note marked fixed
\newcommand\fixednote[1]{}
\newcommand\discuss[1]{\mnote{ #1 \textcolor{red}{discuss}}}
\newcommand\reword[1]{\textcolor{orange}{ #1 }}

\newcommand\lang{ILD }

\title[Title]{\bf A minimal functional programming language designed for easy partial evaluation}

\author[Angel]{Angel Angelov}
% \address{Baba \\
% foo
% }
\setlength{\parindent}{0pt}
\email{angel@qtrp.org}

% \date{\today}

\newtheorem{theorem}{Theorem}[section]
\newtheorem{lemma}[theorem]{Lemma}
\newtheorem{corollary}[theorem]{Corollary}
\newtheorem{proposition}[theorem]{Proposition}
\newtheorem{conjecture}[theorem]{Conjecture}
\theoremstyle{definition}
\newtheorem{defn}[theorem]{Definition}
\newtheorem{example}[theorem]{Example}
\newtheorem{remark}[theorem]{Remark}

\begin{document}

\maketitle

\begin{center}
Faculty of Mathematics and Informatics, Sofia University
\end{center}

\section*{Abstract}

I present a programming language called \lang that loosely resembles Lisp
and Scheme, having the following distinctive features:

\begin{enumerate}
    \item \lang is very small: the core language has only two special
    forms (\texttt{quote}, \texttt{expand})
    \item \lang is expressive enough that common programming constructs may
    be implemented via metaprogramming
    \item \lang supports first-class continuations (like Scheme)
    \item\label{immutable} Basic data structures in \lang are immutable
    \item\label{isolatedsiblings} While evaluating a program, the value of an expression does not
    depend on the values of its siblings
    \item\label{runtimemacros} Lisp uses two-phase approach for evaluating macros and code,
    while in \lang calling a macro is done in the same context as
    calling a regular function.
\end{enumerate}

\lang is a research language meant to facillitate experiments with partial evaluation.
The goal of further research is to determine whether partial evaluation might be
a viable strategy for eliminating some of the overhead caused by extensive metaprogramming,
making such a minimal language performant enough for real-world use.



% \begin{thebibliography}{99}
%
% \bibitem{BS} E. Bannai, N. J. A. Sloane, Uniqueness of certain spherical codes, {\it Can. J. Math.} 33, 1981, 437-449.
%
% \bibitem{CS} {	J.H.Conway, N.J.A.Sloane}, {\it Sphere Packings, Lattices and Groups}, Springer -- Verlag, New York 1988.
%
% \bibitem{Sze}
% G.\,Szeg\H{o}, \emph{Orthogonal Polynomials},
% Amer. Math. Soc. Col. Publ., { 23}, Providence, RI, 1939.
%
% \end{thebibliography}

\bigskip
\end{document}
