\documentclass[11pt,a4paper]{amsart}

\usepackage{iftex}
\ifPDFTeX
	\usepackage{mathtext}
	\usepackage[T1,T2A]{fontenc}
	\usepackage[utf8x]{inputenc}
	\usepackage{amssymb}

    \PrerenderUnicode{абвгдежзийклмнопрстуфхцчшщъьюяѝАБВГДЕЖЗИЙКЛМНОПРСТУФХЦЧШЩЪЬЮЯ}
\else % if luatex or xetex
	\usepackage{unicode-math} % this also loads fontspec
	\defaultfontfeatures{Scale=MatchLowercase}
	\defaultfontfeatures[\rmfamily]{Ligatures=TeX,Scale=1}
	\IfFontExistsTF{Stix Two Text}{%
		\setmainfont[]{Stix Two Text}
	}{%
		\setmainfont[]{Noto Serif}
	}
	\IfFontExistsTF{DroidSansM Nerd Font}{%
		\setmonofont[]{DroidSansM Nerd Font}
	}{%
		\setmonofont[]{Droid Sans Mono}
	}
\fi
\usepackage[english,bulgarian]{babel}
\usepackage{amsmath,amsfonts,amsthm}
\usepackage[text={150mm,210mm},centering]{geometry}
\usepackage{xspace}

\newcommand\ds{\displaystyle}
\newcommand\etal{et al.\xspace}
\makeatletter
\def\@@and{и}
\makeatother

\title{\MakeUppercase{експериментален език за програмиране, насочен към лесно частично оценяване}}


\author[angel]{Ангел Ангелов}
\email{angel@qtrp.org}

\newtheorem{theorem}{Теорема}%[section]
\newtheorem{lemma}[theorem]{Лема}
\newtheorem{corollary}[theorem]{Следствие}
\newtheorem{proposition}[theorem]{Твърдение}
\newtheorem{conjecture}[theorem]{Предположение}
\theoremstyle{definition}
\newtheorem{defn}[theorem]{Дефиниция}
\newtheorem{example}[theorem]{Пример}
\newtheorem{remark}[theorem]{Забележка}

\begin{document}
\newcommand\FMI{Факултет по математика и информатика,
Софийски университет "`Св. Климент\par Охридски"',
бул. "`Джеймс Баучър"' \No~5, 1164 София, България}

\newcommand\NSF{Фонд "`Научни изследвания"' при МОН\xspace}

\newcommand\SUF{Фонд "`Научни изследвания"' при СУ "`Св. Климент Охридски"'\xspace}

\renewcommand\emailaddrname{{\itshape E-mail}}

\maketitle
\begin{center}
\FMI
\end{center}

\thispagestyle{empty}

\bigskip

\section*{Резюме%
%\\{\small\normalfont (Тази работа ще бъде представена от Ангел Ангелов)}
}

\bigskip




\bigskip
Този шаблон ще Ви помогне да форматирате Вашето резюме.
Моля въведете текста, запазвайки формата и стиловете.
Цитиране на литературни източници не е задължително.

Примери:

\begin{theorem}\label{th:1}
Текст на теоремата.
\end{theorem}

\begin{defn}\label{def:1}
Текст на дефиницията.
\end{defn}

Виж~\cite{BS,Tagi} и Теорема~\ref{th:1}.

\begin{thebibliography}{99}

\bibitem{BS} E. Bannai, N.\,J.\,A. Sloane,
Uniqueness of certain spherical codes,
\emph{Can. J. Math.} 33, 1981, 437--449.

\bibitem{CS}
J.\,H. Conway, N.\,J.\,A. Sloane,
\emph{Sphere Packings, Lattices and Groups}, New York, Springer-Verlag, 1988.

\bibitem{Tagi}
Я. Тагамлицки,
\emph{Диференциално смятане},
V изд.,
София, Наука и изкуство, 1971.

\end{thebibliography}

\end{document}
