\documentclass[11pt,a4paper]{amsart}

\usepackage{iftex}
\ifPDFTeX
	\usepackage{mathtext}
	\usepackage[T1,T2A]{fontenc}
	\usepackage[utf8x]{inputenc}
	\usepackage{amssymb}

    \PrerenderUnicode{абвгдежзийклмнопрстуфхцчшщъьюяѝАБВГДЕЖЗИЙКЛМНОПРСТУФХЦЧШЩЪЬЮЯ}
\else % if luatex or xetex
	\usepackage{unicode-math} % this also loads fontspec
	\defaultfontfeatures{Scale=MatchLowercase}
	\defaultfontfeatures[\rmfamily]{Ligatures=TeX,Scale=1}
	\IfFontExistsTF{Stix Two Text}{%
		\setmainfont[]{Stix Two Text}
	}{%
		\setmainfont[]{Noto Serif}
	}
	\IfFontExistsTF{DroidSansM Nerd Font}{%
		\setmonofont[]{DroidSansM Nerd Font}
	}{%
		\setmonofont[]{Droid Sans Mono}
	}
\fi
\usepackage[english,bulgarian]{babel}
\usepackage{amsmath,amsfonts,amsthm}
\usepackage[text={150mm,210mm},centering]{geometry}
\usepackage{xspace}
\setlength{\parindent}{0pt}

\newcommand\ds{\displaystyle}
\newcommand\etal{et al.\xspace}
\makeatletter
\def\@@and{и}
\makeatother

\title{\MakeUppercase{експериментален език за програмиране, насочен към лесно частично оценяване}}


\author[angel]{Ангел Ангелов}
\email{angel@qtrp.org}

\newtheorem{theorem}{Теорема}%[section]
\newtheorem{lemma}[theorem]{Лема}
\newtheorem{corollary}[theorem]{Следствие}
\newtheorem{proposition}[theorem]{Твърдение}
\newtheorem{conjecture}[theorem]{Предположение}
\theoremstyle{definition}
\newtheorem{defn}[theorem]{Дефиниция}
\newtheorem{example}[theorem]{Пример}
\newtheorem{remark}[theorem]{Забележка}

\begin{document}
\newcommand\FMI{Факултет по математика и информатика,
Софийски университет "`Св. Климент\par Охридски"',
бул. "`Джеймс Баучър"' \No~5, 1164 София, България}

\newcommand\NSF{Фонд "`Научни изследвания"' при МОН\xspace}

\newcommand\SUF{Фонд "`Научни изследвания"' при СУ "`Св. Климент Охридски"'\xspace}

\renewcommand\emailaddrname{{\itshape E-mail}}

\newcommand\lang{ILD }

\maketitle
\begin{center}
\FMI
\end{center}

\thispagestyle{empty}

\bigskip

\section*{Резюме%
%\\{\small\normalfont (Тази работа ще бъде представена от Ангел Ангелов)}
}

\bigskip

В този доклад представям език за програмиране \lang, който е подобен на
езиците LISP и Scheme и има следните отличителни признаци:

\begin{enumerate}
    \item \lang е малък: единствените специални форми, дефинирани в основния
    език са \texttt{quote} и \texttt{expand}
    \item \lang е достатъчно изразителен за да може често-срещани в програмирането
    конструкции да бъдат имплементирани чрез метапрограмиране
    \item \lang поддържа continuations като първокласен обект (подобно на Scheme)
    \item\label{immutable} Основните структури от данни в \lang не позволяват промяна (mutability)
    \item\label{isolatedsiblings} По време на оценяване на програма, оценката на произволен
    израз от нея не зависи от неговите съседни изрази
    \item\label{runtimemacros} За разлика от LISP, където макросите и основният код
    се оценяват в две отделни фази, в \lang макросите се оценяват в същия контекст,
    в който и обикновените функции.
\end{enumerate}

\lang е учебен език, замислен да позволява лесни експерименти с частично оценяване.
Целта на по-нататъчните изследвания е да се оцени възможността за използване
на частично оценяване като стратегия за елиминиране на част от допълнителните изчисления,
причинени от голямото количество метапрограмиране до степен, която би направила
такъв минимален език пригоден за използване върху реални задачи.

\bigskip
\end{document}
