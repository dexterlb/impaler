\documentclass[11pt,a4paper]{amsart}

\usepackage{iftex}
\ifPDFTeX
	\usepackage{mathtext}
	\usepackage[T1,T2A]{fontenc}
	\usepackage[utf8x]{inputenc}
	\usepackage{amssymb}

    \PrerenderUnicode{абвгдежзийклмнопрстуфхцчшщъьюяѝАБВГДЕЖЗИЙКЛМНОПРСТУФХЦЧШЩЪЬЮЯ}
\else % if luatex or xetex
	\usepackage{unicode-math} % this also loads fontspec
	\defaultfontfeatures{Scale=MatchLowercase}
	\defaultfontfeatures[\rmfamily]{Ligatures=TeX,Scale=1}
	\IfFontExistsTF{Stix Two Text}{%
		\setmainfont[]{Stix Two Text}
	}{%
		\setmainfont[]{Noto Serif}
	}
	\IfFontExistsTF{DroidSansM Nerd Font}{%
		\setmonofont[]{DroidSansM Nerd Font}
	}{%
		\setmonofont[]{Droid Sans Mono}
	}
\fi
\usepackage[english,bulgarian]{babel}
\usepackage{amsmath,amsfonts,amsthm}
\usepackage[text={150mm,210mm},centering]{geometry}
\usepackage{xspace}
\setlength{\parindent}{0pt}

\newcommand\ds{\displaystyle}
\newcommand\etal{et al.\xspace}
\makeatletter
\def\@@and{и}
\makeatother

\title{\MakeUppercase{експериментален език за програмиране, насочен към лесно частично оценяване}}


\author[angel]{Ангел Ангелов}
\email{angel@qtrp.org}

\newtheorem{theorem}{Теорема}%[section]
\newtheorem{lemma}[theorem]{Лема}
\newtheorem{corollary}[theorem]{Следствие}
\newtheorem{proposition}[theorem]{Твърдение}
\newtheorem{conjecture}[theorem]{Предположение}
\theoremstyle{definition}
\newtheorem{defn}[theorem]{Дефиниция}
\newtheorem{example}[theorem]{Пример}
\newtheorem{remark}[theorem]{Забележка}

\begin{document}
\newcommand\FMI{Факултет по математика и информатика,
Софийски университет "`Св. Климент\par Охридски"',
бул. "`Джеймс Баучър"' \No~5, 1164 София, България}

\newcommand\NSF{Фонд "`Научни изследвания"' при МОН\xspace}

\newcommand\SUF{Фонд "`Научни изследвания"' при СУ "`Св. Климент Охридски"'\xspace}

\renewcommand\emailaddrname{{\itshape E-mail}}

\newcommand\lang{ILD }

\maketitle
\begin{center}
\FMI
\end{center}

\thispagestyle{empty}

\bigskip

\section*{Резюме%
%\\{\small\normalfont (Тази работа ще бъде представена от Ангел Ангелов)}
}

\bigskip

В този доклад представяме езика за програмиране \lang, който е подобен на
езиците LISP и Scheme и има следните отличителни признаци:

\begin{enumerate}
    \item минимален синтаксис: единствените специални форми, дефинирани в основния
    език, са \texttt{quote} и \texttt{expand};
    \item достатъчна изразителност, позволяваща имплементирането на често срещани програмни
    конструкции чрез метапрограмиране;
    \item поддържка на продължения (continuations) като първокласни обекти, подобно на Scheme;
    \item\label{immutable} основните структури от данни са неизменими (immutable);
    \item\label{isolatedsiblings} изолация: оценката на произволен
    израз не зависи от неговите съседни изрази;
    \item\label{runtimemacros} макросите се оценяват в същия контекст, в който и обикновените
    функции, за разлика от LISP, където макросите и основният код се оценяват в две отделни фази.
\end{enumerate}

\lang е език, създаден специално за научни експерименти с частично оценяване.
Целта на по-нататъшните изследвания е да се оцени възможността за използване
на частично оценяване като стратегия за елиминиране на част от допълнителните изчисления,
причинени от голямото количество метапрограмиране, до степен, която би направила
такъв минимален език пригоден за използване върху реални задачи.

\bigskip
\end{document}
