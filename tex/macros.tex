\usepackage{fontenc}
\usepackage{fontspec}
\usepackage{libertine}

% \newfontfamily\fontcomic[NFSSFamily=comic]{Comic Sans MS}

\defaultfontfeatures{Ligatures=TeX}

\usepackage{indentfirst}
\usepackage{url}
\usepackage{color}
\usepackage{xcolor}
\usepackage{hhline}
\usepackage{xspace}
\usepackage{minibox}
\usepackage{pbox}
\usepackage{mathtools}
\usepackage[framemethod=TikZ]{mdframed}
\usepackage{amsthm}
\usepackage{amssymb}
\usepackage{amsmath}
\usepackage[nameinlink]{cleveref}
\usepackage{mathabx}
\usepackage{graphicx}
\usepackage{ebproof}
\usepackage[toc,page]{appendix}
\usepackage{adjustbox}
\usepackage{lstautogobble}
\usepackage{listings}
\usepackage{syntax}
\usepackage[normalem]{ulem}
\usepackage{subfiles}
\usepackage{csquotes}
\usepackage[style=alphabetic]{biblatex}
\usepackage{tikz}
\usetikzlibrary{fit}
\usetikzlibrary{external}
\usetikzlibrary{positioning}
\tikzset{main node/.style={circle,fill=blue!5,draw,minimum size=1cm,inner sep=0pt},
            }
\tikzset{
    external/system call={%
    xelatex \tikzexternalcheckshellescape
    -halt-on-error -interaction=batchmode -shell-escape
    -jobname "\image" "\texsource"}}
% \tikzexternalize

% function arrow
\newcommand\fun{\rightarrow}
% partial function arrow
\newcommand\pfun{\mathrel{\ooalign{\hfil$\mapstochar$\hfil\cr$\to$\cr}}}

\newcommand\code[1]{\mbox{\texttt{#1}}}
\newcommand\cendrow{\vspace{0.4em} \\}

% margin note
\newcommand\mnote[1]{\leavevmode\marginpar{\scriptsize{ #1 }}}
\newcommand\note[1]{\mnote{\textcolor{orange}{ #1 }}}
% margin note marked fixed
\newcommand\fixednote[1]{}
\newcommand\discuss[1]{\mnote{ #1 \textcolor{red}{discuss}}}
\newcommand\reword[1]{\textcolor{orange}{ #1 }}

\usepackage{color}
\definecolor{Bluish}{rgb}{0.39,0.55,0.78}
\definecolor{exampleness}{HTML}{ebf9d6}
\definecolor{codeness}{HTML}{fbf1c7}
\definecolor{commentness}{HTML}{b57614}
\definecolor{light-gray}{gray}{0.9}
\definecolor{celadon}{rgb}{0.67, 0.88, 0.69}
\definecolor{asparagus}{rgb}{0.53, 0.66, 0.42}
\definecolor{linkness}{rgb}{0.12, 0.43, 0.41}

% fixme note
\newcommand\fixme[1]{
    \greenbox{
        \vspace{1em}
        \textbf{\textcolor{red}{FIXME:} } #1
        \vspace{1em}
    }
}

% some settings
\DeclareRobustCommand{\hsout}[1]{\texorpdfstring{\sout{#1}}{#1}}

\renewcommand{\baselinestretch}{1.1}
\setlength{\emergencystretch}{3em}
\setlength{\parskip}{5pt}
\setlength{\parindent}{0pt}

\appto{\bibsetup}{\sloppy}
\addbibresource{references.bib}

\graphicspath{ {./images/} }

% boxes
\tikzset{%
  boxcola/.style={rectangle,rounded corners,fill=asparagus,draw=asparagus,fill opacity=0.02,thick,inner sep=5pt}
}
\newcommand\boxtexta{asparagus}

\tikzset{%
  boxcolb/.style={rectangle,rounded corners,fill=blue,draw=blue,fill opacity=0.02,thick,inner sep=5pt}
}
\newcommand\boxtextb{blue}

\newcommand\greenbox[1]{
    \tikzexternaldisable
    \begin{samepage}
        \begin{mdframed}[%
            backgroundcolor=green!8,
            linecolor=gray!50!green!60,
            outerlinewidth=0.5pt,
            roundcorner=5mm,
            skipabove=\baselineskip,
            skipbelow=\baselineskip,
            leftmargin=1cm,
            rightmargin=1cm,
        ]
            #1
        \end{mdframed}
    \end{samepage}
    \tikzexternalenable
}

\newenvironment{mexample}
    {
        \tikzexternaldisable
        \begin{mdframed}[%
            backgroundcolor=exampleness,
            rightline=false,
            topline=false,
            bottomline=false,
            leftline=false,
        ]
            \begin{examplethm}
    }
    {
            \end{examplethm}
        \end{mdframed}
        \tikzexternalenable
    }

\newcommand{\cfigure}[1]{
    \begin{figure}[htbp!]
        \centering
        #1
    \end{figure}
}
\numberwithin{equation}{section}
\numberwithin{figure}{section}
\numberwithin{table}{section}

\newenvironment{lstwrap}
    {
    }
    {
    }

\lstset{
	backgroundcolor = \color{codeness},
    autogobble,
    columns=fixed,
    showspaces=false,
    showtabs=false,
    breaklines=true,
    showstringspaces=false,
    breakatwhitespace=true,
    escapeinside={(*@}{@*)},
    language=Haskell,
    commentstyle=\color{commentness},
    keywordstyle=\color{black},
    stringstyle=\color{black},
    numberstyle=\color{black},
    basicstyle=\ttfamily\footnotesize,
    frame=lrtb,
    framesep=12pt,
    xleftmargin=12pt,
    xrightmargin=12pt,
    tabsize=4,
    captionpos=b,
    literate={symlambda}{{$\lambda$}}1 {symabstr}{{$\abstr$}}1 {symtot}{{$\tot$}}1
}

% restriction symbol
\newcommand\restr[2]{{% we make the whole thing an ordinary symbol
  \left.\kern-\nulldelimiterspace % automatically resize the bar with \right
  #1 % the function
  \vphantom{\big|} % pretend it's a little taller at normal size
  \right|_{#2} % this is the delimiter
  }}

\hypersetup{
    colorlinks=true,
    linktoc=all,
    citecolor=linkness,
    filecolor=linkness,
    linkcolor=linkness,
    urlcolor=linkness
}

\newcommand\printtoc{
    {
        \hypersetup{linkcolor=black}
        \tableofcontents
    }
}

\usepackage{tabularx}


\renewcommand{\land}{\mathbin{\&}}
\newcommand{\xor}{\veebar}
\newcommand{\boolset}{\mathbb{B}}

\newcommand{\latex}{\LaTeX\xspace}

\newcommand{\pika}{\includegraphics[height=15pt]{images/pika.png}}

\newenvironment{mitemize}
    {
        \begin{itemize}
        \setlength{\itemsep}{0em}
    }
    {
        \end{itemize}
    }

\newenvironment{menumerate}
    {
        \begin{enumerate}
        \setlength{\itemsep}{0em}
    }
    {
        \end{enumerate}
    }

\newcommand{\menumref}[1]{(\ref{#1})}
