\begin{document}

\begin{frame}
\maketitle
\end{frame}

\begin{frame}[fragile]{Goals of the \lang language}
  \begin{itemize}
    \item A Lisp (in particular, a Lisp1)
    \item Minimal core language
      \begin{itemize}
        \item Only two special forms: \verb|quote|, \verb|expand| and a macro expansion operator
        \item Implement everything else in the language itself
      \end{itemize}
    \item Expressive
      \begin{itemize}
        \item Classic features found in Lisp and Scheme: closures, macros, continuation-passing style
      \end{itemize}
  \end{itemize}
\end{frame}

\begin{frame}[fragile]{Goals}
\begin{itemize}
    \item Precise control over compile-time vs runtime execution
        \begin{itemize}
            \item Macro evaluation has ``runtime'' semantics: the macro body can make use of values defined in the runtime environment
            \item Constant data, including macro execution, may be evaluated at compile time during PE
        \end{itemize}
    \item Isolation
        \begin{itemize}
            \item The value of an expression does not depend on its sibling expressions
        \end{itemize}
\end{itemize}
\end{frame}

\begin{frame}[fragile]{Example code in \lang}
\begin{minted}{lisp}
(module
  (doc "this module calculates factorials")
  (exports main)
  (imports
    (builtin (mul add <= expand lambda quote))
    ("core/prelude.l" (if const fn)))
  (defs
    (!const main (!lambda () (fact 5)))

    (!fn (fact x)
      (!if (<= x 0)
        1
        (mul x (fact (add x -1)))))))
\end{minted}
\end{frame}

\begin{frame}[fragile]{The macro expansion operator}
\begin{itemize}
    \item This code:
      \begin{minted}{lisp}
        (!if (= x 5)
          "x is 5"
          "x is not 5")
      \end{minted}
    \item Is syntax sugar for this code:
      \begin{minted}{lisp}
        (expand (if (quote (= x 5))
          (quote "x is 5")
          (quote "x is not 5")))
      \end{minted}
    \item Macros are just functions that operate on code.
\end{itemize}
\end{frame}

\begin{frame}[fragile]{Runtime macro execution semantics}
  \begin{minipage}[t]{0.48\textwidth}
    \textbf{\lang}:

    \vspace{0.5em}
    Macro definition:
    \begin{minted}{lisp}
      (!fn (if cnd then else)
        (list
          (list 'bool-to-k cnd)
          then else))
    \end{minted}
    \vspace{0.5em}
    Usage:
    \begin{minted}{lisp}
      (!if (= x 5)
        "x is 5"
        "x is not 5")
    \end{minted}
    \vspace{0.5em}
    Which is the same as:
    \begin{minted}{lisp}
      (expand (if (quote (= x 5))
        (quote "x is 5")
        (quote "x is not 5")))
    \end{minted}
    \vspace{0.5em}
    Which evaluates to the result of:
    \begin{minted}{lisp}
      ((bool-to-k (= x 5))
        "x is 5"
        "x is not 5")
    \end{minted}
  \end{minipage}
  \begin{minipage}[t]{0.48\textwidth}
    \textbf{Lisp}:

    \vspace{0.5em}
    Macro definition:
    \begin{minted}{lisp}
      (defmacro if (cnd then else)
        (list
          (list 'bool-to-k cnd)
          then else))
    \end{minted}

    \vspace{0.5em}
    Usage:
    \begin{minted}{lisp}
      (if (= x 5)
        "x is 5"
        "x is not 5")
    \end{minted}

    \vspace{0.5em}
    Gets replaced at compile time by:
    \begin{minted}{lisp}
      ((bool-to-k (= x 5))
        "x is 5"
        "x is not 5")
    \end{minted}

    \vspace{0.5em}
    Which gets evaluated at runtime
  \end{minipage}
\end{frame}

\begin{frame}[fragile]{Runtime macro execution semantics}
  \begin{minipage}[t]{0.48\textwidth}
    \textbf{Pros:}

    \begin{itemize}
      \item Macros are more powerful (they see local bindings
      in addition to global ones)
      \item No extra macro expansion pass
      \begin{itemize}
        \item Makes it trivial to write a basic interpreter
      \end{itemize}
    \end{itemize}
  \end{minipage}
  \begin{minipage}[t]{0.48\textwidth}
    \textbf{Cons:}

    \begin{itemize}
      \item Resolving metaprogramming at runtime carries
      a performance penalty
      \item Resolving nested macros may be counterintuitive
    \end{itemize}
  \end{minipage}

  \vspace{2em}

  PE lets us address the runtime performance penalty "for free",
  because the argument of a macro is always a constant.
\end{frame}

\begin{frame}[fragile]{Embedding}
\begin{itemize}
    \item In \lang:
        \begin{itemize}
            \item Application software
            \item Libraries
            \item DSL definitions (syntactic constructs, etc)
        \end{itemize}
    \item In the host language:
        \begin{itemize}
            \item IO
            \item Data structures that depend on memory layout
            \item Scheduling (tasks, async IO, threads)
            \item Garbage collection
        \end{itemize}
\end{itemize}
\end{frame}

% \begin{frame}[fragile]{Example: error handling}
%
% \begin{minted}{lisp}
% (defs
%     ...
%     (!fn (e-map f l)
%         (!propagate-err-with try
%             (!if (empty? l)
%                 l
%                 (cons (try (f (car l))) (e-map f (cdr l)))))))
% \end{minted}
%
% \end{frame}

\begin{frame}[fragile]{Ongoing work}
\begin{todolist}
  \item[\done] Proof of concept interpreter
  \item[\done] Basic partial evaluator
  \item Non-diverging partial evaluator
  \item Practical libraries
  \item Optimising compiler
\end{todolist}
\end{frame}

\begin{frame}[fragile,plain]
\begin{center}
\Huge Thank you!
\end{center}
\end{frame}
\end{document}
